\vspace{-4mm}
\section{Current and Future Work}

Policy automata and \tfppf~are just the first step towards a complete solution for evolving privacy policies. While these approaches provide support for a whole new family of privacy policies they still have some limitations. For instance, consider that Alice enables the following policy \emph{``Only my friends can see my pictures during the weekend''}. Imagine that Alice and Bob are not friends. If Alice shares a picture on Saturday, Bob will not have access to it. However, on Monday this policy would be deactivated. What would be the effect of turning off this policy? It might be possible that Bob gains access to all the pictures that Alice posted during the weekend, since no restrictions are specified outside the scope of the weekend. In order to address this problem we might need a policy language able to express {\it real-time} aspects. For \tfppf~we are currently extending the logic with timestamps in predicates and modalities. It gives the needed expressive power to protect some piece of information shared at a given moment in time. For policy automata it is bit more complicated, since the approach is policy language agnostic. One way for policy automata to solve this problem is being aware of the timestamp of the pieces of information it is protecting (or monitoring). Combining the real-time extension of \tfppf~and policy automata might result in a feasible solution. We are studying how to adapt policy automata to a policy language that contains timestamps indicating when some information has been shared.

Furthermore, the connection between policy automata and \tfppf~is an interesting question even in their current form. \tfppf~does not have an enforcement mechanism, so it can be seen as a language suitable for reasoning and specification of privacy policies. On the other hand, policy automata are enforceable through \larva~monitors, but it is not so targeted for policy reasoning. We plan to investigate what is the connection between the two formalisms and how to convert \tfppf~policies into policy automata, thus providing an enforcement mechanism for \tfppf.
\vspace{-4mm}
\paragraph{Related work.} To the best of our knowledge we are the first to offer support for specification and enforcement of evolving privacy policies. However, the techniques we use has extensively been used in the security literature. Epistemic logic has been used in the form of interpreted systems (which include temporal operators as the ones used in \tfppf) to describe access control and information flow properties in multiagent systems~\cite{FHM+95rk,HK08sms,B13lifadp,RT11lkfsn}. Moreover, we are currently showing the correspondance between traditional Kripke semantics and the semantics based on SNMs for \fppf. As for policy automata, there is work done in the context of security policies, for instance the work by Le Guernic \etal~on using automata to monitor and enforce non-interference~\cite{LeGuernic2007,LeGuernic07acmcp} or by Schneider on security automata \cite{Schneider00esp}. It could be instructive to further develop the theoretical foundations of policy automata and relate it to security automata and their successors (e.g., edit automata \cite{LBW05ea}).
